\documentclass[letterpaper,11pt]{article}
\setlength{\textwidth}{6.5in}
\setlength{\hoffset}{0in}
%TODO: tweak these a bit more
\setlength{\voffset}{-0.5in}
\setlength{\textheight}{8.75in}
\setlength{\marginparsep}{0in}
\setlength{\marginparwidth}{0in}
\setlength{\oddsidemargin}{0in}

\usepackage{aas_macros}
\usepackage{amssymb}
\usepackage{hyperref}
\usepackage{listings}
\usepackage{mathabx}

\usepackage{graphicx}
\usepackage{lineno}
\usepackage[authoryear,square,colon]{natbib}
\bibpunct{[}{]}{;}{a}{,}{,~}

\usepackage{lipsum}

\begin{document}


\title{Searching for the $E^{-3/2}$ Suprathermal Ion Tail in Parker Solar Probe's IS$\Sun$IS Data}
\author{A. Merrill}
\date{\today}
\maketitle

\begin{abstract}
\lipsum[1]
\end{abstract}

\linenumbers

\section{Introduction}
\label{sec:intro}
A surprising result of the \textit{Advanced Composition Explorer} (ACE) and the \textit{Ulysses} mission is the seemingly ever-present power-law tail in ion distributions of solar wind, even in the absence of solar activity~\citep{Gloeckler_2000}.  These populations of high energy particles are of particular importance to humanity as they can serve to seed possibly destructive high energy events.  Such events are a direct hazard to humans and machines in space, and in extreme circumstances can even pose threats to ground-based equipment.  The shape of the suprathermal tail can also inform us about other accelerating processes in the solar wind which can cause similar threats to humanity.

Parker Solar Probe's IS$\Sun$IS EPILo instrument provides never before seen energy resolutions at regimes closer to the sun than any previous spacecraft.  These new measurements allow us to probe for the power-law tail in new regimes, providing insight into the phenomena and mechanisms that create and accelerate these high energy particles.

\section{A Bunch of Sections to Follow}
\lipsum[2]

\appendix
\section{First Section of Appendix}
\lipsum[3]

\bibliographystyle{thesis}
\bibliography{thesis}

\end{document}
