\documentclass[letterpaper,11pt]{article}
\setlength{\textwidth}{6.5in}
\setlength{\hoffset}{0in}
%TODO: tweak these a bit more
\setlength{\voffset}{-0.5in}
\setlength{\textheight}{8.75in}
\setlength{\marginparsep}{0in}
\setlength{\marginparwidth}{0in}
\setlength{\oddsidemargin}{0in}

\usepackage{aas_macros}
\usepackage{amssymb}
\usepackage{hyperref}
\usepackage{listings}
\usepackage{mathabx}

\usepackage{graphicx}
\usepackage{lineno}
\usepackage[authoryear,square,colon]{natbib}
\bibpunct{[}{]}{;}{a}{,}{,~}

\usepackage{lipsum}

\begin{document}


\title{Searching for the $E^{-3/2}$ Suprathermal Ion Tail in Parker Solar Probe's IS$\Sun$IS Data}
\author{A. Merrill}
\date{\today}
\maketitle

\begin{abstract}
\lipsum[1]
\end{abstract}

%\linenumbers

\section{Introduction}
\label{sec:intro}
A question of frequent interest across many fields is the possible
relationship between two types of observation, potentially with some
time delay. Several familiar tools, such as regression analysis and
cross-correlations, relate continuously varying quantities. We expect,
for example, a strong correlation between the time series of
temperatures in White Rock and Los Alamos. A somewhat weaker
correlation would be evident between Los Alamos and Oklahoma
City, with Los Alamos expected to lag by about half an hour in the
dominant diurnal variation.

Less familiar are tools for associating point events, rather than
continuously varying quantities. For instance we may be interested in
a possible association between hailstorms in Los Alamos and work
requests at body shops in northern New Mexico. Although potentially
reducible to pseudo-continuous quantities (e.g., storms per month and
body shop jobs per month), this reduction loses the temporal
association between individual events and may require a period of
averaging longer than the delay between types of events. This report
reviews and extends techniques for determining the level of
association between two types of event, any delay in the association,
and a confidence that the observed level of association is significant
(beyond that expected from chance.) These techniques are agnostic
with respect to causality and work even if only a subset of each event
class is associated with the other: there may be reasons other than
hailstorms to seek body shop work. Subject to the caveats in
section~\ref{sec:ci}, they also work for clustered events,
e.g., multiple requests for body work following a single hailstorm.

This report builds on methods presented in the space sciences by
\citet{2007GeoRL..3408104M}, who examined the association between
magnetospheric substorms and ``trigger events'' in the upstream solar
wind. Python code implementing the methods of this report is available
in the LANL SpacePy package \citep[LA-CC-10-064;][]{spacepy11},
available under an open source license from
\url{http://spacepy.lanl.gov/}. These techniques are applicable to a
range of scientific studies.

\section{A Bunch of Sections to Follow}
\lipsum

\appendix
\section{First Section of Appendix}
\lipsum

\bibliographystyle{thesis}
\bibliography{thesis}

\end{document}
